%& -job-name=Assignment5
\input{./Documents-Common/preamble.tex}
\begin{document}
  \maketitle[Computational Finance]{Pricing of American Options}
  In this report we discuss the pricing of American Options
  \section{Monte Carlo}
In order to price American options using Monte Carlo one simulates several price paths under a financial model (e.g. Black Scholes) similar to how one does for European options.
However one has to take into account variable stopping time $\tau$, ($0 \leq \tau \leq T$), the value of an american option at $t=0$ is given by
\begin{align}
  \label{eq:mc_raw}
  V(S, 0) = \sup_{0\leq \tau \leq T} E(e^{-r\tau}\Psi(S_\tau)|S_0 = S)
\end{align}.
That is, the supremum w.r.t. time of the expected, discounted, pay-off. (So simply, the regular value at the best time $\tau$).
We approximate \eqref{eq:mc_raw} by computing a lower- and upper-bound on the supremum.
that is
\begin{align}
  V^\text{low}(S,0) \leq V(S,0) \leq V^\text{up}(S,0)
\end{align}
\par In order to compute the lower bound we simply have to decide upon a stopping strategy and simulate it (without looking farther into the future).
One such strategy would be to define a curve in $(s,t)$ space and exercise as soon as we pass this curve. The simplest curve is of course a fixed value of $S = \beta$.
while we can pick any strategy we wish, it is best to pick best possible strategy to try and get the spread between $V^\text{low}$ and $V^\text{up}$ as small as possible (and make as much profit as possible using said strategy).
\par In order to compute an upper bound, we simulate the value of $V$ for all values of $t$ and then pick $\tau$ to be the ideal time to exercise.
  \section{Finite Differences}
A finite difference approach to American Options leads to an open boundary equation, where one of the boundaries (left or right, depending on the type of option) is given by
the early exercise curve. The derivation of this is given in Seydel 4.5 to 4.6.1 (with the back-transform descibed in algorithm 4.14). We will not reproduce the full derivation for brevities sake but do note that we end up with a matrix equation
similar to what we find for European options only this time with additional restrictions, in the notation of Seydel:
\begin{align}
  Aw-b \geq 0,\nn
  w \geq g,\nn
  (Aw-b)^T(w-g) = 0.
\end{align}
This is a Linear Complementarity Problem (LCP) and like matrix equations, there exist algorithms for solving this system for $w$. One such method is the iterative {\em Projected Succesive Over-Relaxation} (PSOR), and if $A$ is tri-diagonal we can also use the Brennan-Schwarz Algorithm to solve it directly, similar to how we could do so for normal matrix equations and tridiagonal matrices.
\par Knowing this, the general structure of the problem is similar to that of a European option.
\begin{itemize}
  \item{Initialize the necessary variables}
  \item{Loop over each time-step} and solve the LCP-problem for each time-step.
  \item{Convert back to 'normal' coordinates} using the steps described in algorithm 4.14 and then check for early-exercise by computing $S_i_f$.
\end{itemize}
   \section{Operator Splitting}
   Operator splitting has at its base a finite difference approximation as before. However, instead of solving the LCP for $V^i$  one solves for an intermediate variable
   $\hat V^i$ and then projecting the solution so that it fits the constraints. For Crank-Nicolson this means that instead of solving the LCP system we perform the steps
   \begin{align}
     \frac{1}{\Delta t}\left(v^{k+1}-\hat v^k\right) + A \left((1-\alpha)v^{k+1}+\alpha\hat v^k\right)-\lambda^{k+1} = 0,
   \end{align}
   and
   \begin{align}
     \frac{1}{\Delta t}\left(\hat v^k-v^k\right)+ \lambda^{k+1}-\lambda^k &= 0,\nn
     \left[v_i^k-(E-x_i)]\cdot\lambda_i^k &= 0,\nn
     v_i^k &\geq E - x_i,\nn
     \lambda_i^k &\geq 0.
   \end{align}
   The first equation is a normal system of equations for $\hat v^k$ and the second equation can be quickly solved for $\lambda^k$ and $v^k$.
   \par This method is much faster than the LCP method as is shown in the paper by Ikonen and Toivanen.
\end{document}
